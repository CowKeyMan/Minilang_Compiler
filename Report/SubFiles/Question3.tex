\section{Task 3 - XMLGeneration}
\subsection{Explanation and Implementation}

The visitor design pattern is used for the XML generator. Each ASTNode accepts the visitor class's \textit{visit} function through the \textit{accept} function.

An XMLVisitor class was created. This contains a string stream called \textit{xml} which will contain the generated string after the tree is all visited. The \textit{numberOfTabs} integer holds the number of indentation number which should be applied at each line.

\begin{lstlisting}[language=C++]
class XMLVisitor : virtual public Visitor{
	public:
		XMLVisitor(){};
		virtual ~XMLVisitor(){};
		virtual void visit(ASTNode*){};
		virtual void visit(ASTNodeType *n);
		virtual void visit(ASTNodeLiteral *n);
		virtual void visit(ASTNodeIdentifier *n);
		virtual void visit(ASTNodeMultiplicativeOp *n);
		virtual void visit(ASTNodeAdditiveOp *n);
		virtual void visit(ASTNodeRelationalOp *n);
		virtual void visit(ASTNodeActualParams *n);
		virtual void visit(ASTNodeFunctionCall *n);
		virtual void visit(ASTNodeSubExpression *n);
		virtual void visit(ASTNodeUnary *n);
		virtual void visit(ASTNodeFactor *n);
		virtual void visit(ASTNodeTerm *n);
		virtual void visit(ASTNodeSimpleExpression *n);
		virtual void visit(ASTNodeExpression *n);
		virtual void visit(ASTNodeAssignment *n);
		virtual void visit(ASTNodeVariableDecl *n);
		virtual void visit(ASTNodeReturnStatement *n);
		virtual void visit(ASTNodeIfStatement *n);
		virtual void visit(ASTNodeForStatement *n);
		virtual void visit(ASTNodeFormalParam *n);
		virtual void visit(ASTNodeFormalParams *n);
		virtual void visit(ASTNodeFunctionDecl *n);
		virtual void visit(ASTNodeStatement *n);
		virtual void visit(ASTNodeBlock *n);
		virtual void visit(ASTNodeProgram *n);
		void trimXMLNewLines(); // remove empty lines from xml
		string getXML(){ return xml.str(); }
	private:
		stringstream xml;
		unsigned int numberOfTabs = 0;
		string tabsString();
};
\end{lstlisting}

An XML visit for a leaf node accept would look something like this:
\begin{lstlisting}[language=C++]
void XMLVisitor::visit(ASTNodeMultiplicativeOp *n){
	xml << "OP=\"" << n->token->lexeme << "\"";
}
\end{lstlisting}
The above is for a multiplicative operator. So '*' would be shown as 'OP="*"'.
\\\\
This is also a recursive method, so the program node (which contains a list of statements), iteratively goes trough the statements and calls the accept statement for the visitor on them as well, creating recursion.

\begin{lstlisting}[language=C++]
void XMLVisitor::visit(ASTNodeProgram *n){
	for(int i = 0; i < n->statements.size(); ++i){
		n->statements.at(i)->accept(this);
	}
}
\end{lstlisting}

After finishing, the xml string stream can be either outputted to the screen or stored in a separate file.

\subsection{Example Test 1}
Parsing the following as input(parsing as a program node):
\begin{lstlisting}
x = 1+2*4;
\end{lstlisting}

The following output is produced by the xml generator:
\begin{lstlisting}
<Assign>
	x</ID>
	BinExprNode OP="+">
		<IntConst>1</IntConst>
		<BinExpr OP="*">
			<IntConst>2</IntConst>
			<IntConst>4</IntConst>
		</BinExpr>
	</BinExprNode>
</Assign>
\end{lstlisting}
Note how operator precedence is kept.
\\\\
In order to change precedence, enclose the addition in brackets:
\begin{lstlisting}
x = (1+2)*4;
\end{lstlisting}

So the following output is now produced:
\begin{lstlisting}
<Assign>
	x</ID>
	<BinExpr OP="*">
		BinExprNode OP="+">
			<IntConst>1</IntConst>
			<IntConst>2</IntConst>
		</BinExprNode>
		<IntConst>4</IntConst>
	</BinExpr>
</Assign>
\end{lstlisting}

\subsection{Example Test 2}
Putting an entire program as input this time:
\begin{lstlisting}
var x : float = 0;

fn y(g:bool) : int{
	return z;
}

h = g();
\end{lstlisting}

The output by the XML generator is this:
\begin{lstlisting}
<VarDecl>
	<Var Type = "float">x</ID>
	<IntConst>0</IntConst>
</VarDecl>
<FuncDecl>
	<FN Type = "int">y</ID>
	<F_Param> g</ID>:Type = "bool" </F_Param>
	<Return>
		z</ID>
	</Return>
</FuncDecl>
<Assign>
	h</ID>
	<FN_CALL FN=g</ID>"
	</FN_CALL FN>
</Assign>
\end{lstlisting}