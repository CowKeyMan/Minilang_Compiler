\section{Task 5 - Interpretation}
\subsection{Explanation and Implementation}
The interpreter is to execute the program line by line and report any run time errors if there are any. The symbol table is regenerated and this time it will contain identifiers and their contents besides just their types. This content will also be changed as the program executes.

A new visitor class was created called \textit{IVIsitor}, which contains the following items:

\begin{lstlisting}
// Semantic Analysis
class IVIsitor : virtual public Visitor{
	public:
	IVisitor(){};
	virtual ~IVisitor(){};
	virtual void *visit(ASTNode*){ return 0; };
	virtual void *visit(ASTNodeType *n);
	virtual void *visit(ASTNodeLiteral *n);
	virtual void *visit(ASTNodeIdentifier *n);
	virtual void *visit(ASTNodeMultiplicativeOp *n);
	virtual void *visit(ASTNodeAdditiveOp *n);
	virtual void *visit(ASTNodeRelationalOp *n);
	virtual void *visit(ASTNodeActualParams *n);
	virtual void *visit(ASTNodeFunctionCall *n);
	virtual void *visit(ASTNodeSubExpression *n);
	virtual void *visit(ASTNodeUnary *n);
	virtual void *visit(ASTNodeFactor *n);
	virtual void *visit(ASTNodeTerm *n);
	virtual void *visit(ASTNodeSimpleExpression *n);
	virtual void *visit(ASTNodeExpression *n);
	virtual void *visit(ASTNodeAssignment *n);
	virtual void *visit(ASTNodeVariableDecl *n);
	virtual void *visit(ASTNodePrintStatement*);
	virtual void *visit(ASTNodeReturnStatement *n);
	virtual void *visit(ASTNodeIfStatement *n);
	virtual void *visit(ASTNodeForStatement *n);
	virtual void *visit(ASTNodeFormalParam *n);
	virtual void *visit(ASTNodeFormalParams *n);
	virtual void *visit(ASTNodeFunctionDecl *n);
	virtual void *visit(ASTNodeStatement *n);
	virtual void *visit(ASTNodeBlock *n);
	virtual void *visit(ASTNodeProgram *n);
	private:
};
\end{lstlisting}

Next, the functionality of each node is discussed:
\begin{itemize}
	\item Type : void
		\subitem Interpreter ignores this
	\item Literal : Value
		\subitem The value of the literal is returned
	\item Identifier : string (the lexeme of the identifier token)
		\subitem returns name of the identifier. Lookup for value is handled by parent node
	\item MultiplicativeOp : MultOp Type (ex. TIMES, AND)
		\subitem returns the operator itself, so that the expression node it belongs to can use it to operate on the values
	\item AdditiveOp : AddOp Token (ex. PLUS, OR)
		\subitem
	\item RelationalOp : Type (ex. EQQ, ST)
		\subitem
	\item ActualParams : Vector of Type
		\subitem
	\item FunctionCall : Type
		\subitem
		\subitem
		\subitem
	\item SubExpression : Type
		\subitem
	\item Unary : Type
		\subitem
	\item Factor : Type
		\subitem
	\item Term
		\subitem
	\item SimpleExpression : Type
		\subitem
		\subitem
	\item Expression : Type
		\subitem
	\item Assignment : void
		\subitem
		\subitem
	\item VariableDecl : void
		\subitem
		\subitem
	\item PrintStatement : void
		\subitem
	\item ReturnStatement : void
		\subitem
	\item IfStatement : void
		\subitem
		\subitem
		\subitem
		\subitem
		\subitem
	\item ForStatement : void
		\subitem
		\subitem
		\subitem
		\subitem
		\subitem
		\subitem
	\item FormalParam : Type
		\subitem
		\subitem
	\item FormalParams : Vector of Type
		\subitem
	\item FunctionDecl : void
		\subitem
		\subitem
	Add function to symbol table after validating size of symbol table is one (outer scope)
		\subitem
		\subitem
	\item Statement : void
		\subitem
		\subitem
	\item Block
		\subitem
	\item Program
		\subitem
\end{itemize}
